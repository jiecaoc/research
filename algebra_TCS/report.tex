\documentclass[12pt,reqno]{amsart}
\usepackage{fullpage}
\usepackage{amsfonts}
\usepackage{amssymb}
\usepackage{times}
\usepackage{graphicx}
\usepackage{hyperref}
\vfuzz=2pt

% some "funny lines" referred to later:
\newtheorem{thm}{Theorem}[section]
\newtheorem{cor}[thm]{Corollary}
\newtheorem{lem}[thm]{Lemma}
\newtheorem{prop}[thm]{Proposition}
\newtheorem{example}[thm]{Example}
\newtheorem{definition}[thm]{Definition}
{ \theoremstyle{remark}\newtheorem*{remark}{Remark} }


\newcommand{\tex}{{\tt .tex}}

%\newenvironment{definition}[1][Definition]{\begin{trivlist}
%\item[\hskip \labelsep {\bfseries #1}]}{\end{trivlist}}

\begin{document}

\title[Short version of title]{Notes on \\ \small \emph{Statistical Learning Theory}}
% or if you want, simply \title{Title of the article}
\author{Jiecao Chen \\ \tiny jieca001@umn.edu }

\maketitle

\begin{abstract}
Most results of statistical learning theory take the form
of so-called error bounds. This note contains introduction and
brief summary of some key ideas and techniques to obtain those
results. Most results come from \cite{paper1}, \cite{Book2}, \cite{paper3}.
\end{abstract}

\section{Introduction}
The main goal of statistical learning theory is to 
provide a framework for studying the problem of inference,
which includes gaining knowledge, making prediction or 
constructing models from the date set. As it is put into 
a statistical framework, that is there are assumptions
of statistical nature about the underlying phenomena.

V. Vapnik once described why we need theories:
\begin{quote}
	\textit{Nothing is more practical than a good theory}.
\end{quote}

A theory of interference should be able to give the 
formal definition of words such as learning, generalization,
overfitting and also to characterize the performance of 
learning algorithms so that it would eventually be able to
help us to design better learning algorithms.

\textbf{Two goals:} make things more precise and derive new or improved
algorithms

\subsection{Learning and Inference}
The process of inductive inference:
\begin{itemize}
\item Observe a phenomenon
\item Construct a model of that phenomenon
\item Make predictions using this model
\end{itemize}
The task of Machine Learning is to actually \textit{automate} this 
process and the goal of Learning Theory is to \textit{formalize} it.


To make our discussion simpler, in this note, we only consider
a special case of above process which is the supervised learning 
framework for pattern recognition. In this special case, data set
consists of instance-label pairs, where the value of label $\in \{-1, 1\}$. The task of a learning algorithm is to construct a function which is able to predict the corresponding label given the instance. A "Good" learning algorithm will give less errors when do such prediction.

A model that can exactly fit all instance-label pairs in a training
data set may not always be the best model, as we know that the data set may sometime include noise, in such case, a model fitting all the data may give a worse prediction in the unseen data, such phenomenon is usually referred to as \textbf{overfitting}. A way to overcome \textbf{overfitting} is to look for \textbf{regularities}.

If there are many models available, and all of the well fit the data, we normally follow Occam's idea to choose the simplest model, which would more likely to be generalized from the training data to future data. This immediately raises the question of how to 
measure and qualify simplicity of a model.

There is no universal way of measuring simplicity and the 
choice of a specific measure inherently depends on the 
problem at hand.

A formula we believe:
\begin{quote}
	\textit{Generalization = Data + Knowledge}
\end{quote}

Generalization can only come when one adds specific knowledge to 
the data. Each learning algorithm encodes specific knowledge.

\subsection{Assumptions}
Several more precise assumptions that made by the Statistical
Learning Theory Framework:

\begin{itemize}
\item Probabilistic model of the phenomenon (or data generation process). Within this model, the relationship between past and
future observations is that they both are sampled independently
from the same distribution (i.i.d).

\end{itemize}


\section{Formalization}
Consider an input space $\mathcal{X}$ and output space $\mathcal{Y}$. As we only consider our specific case (binary classification), we choose $\mathcal{Y} = {-1, 1}$.

$(X, Y) \in \mathcal{X}\times\mathcal{Y}$
are random variables follow the an unknown distribution $P$.  
We observe a sequence of i.i.d pairs $(X_i, Y_i)$ sampled according to $P$ and the goal is to construct a function
$g: \mathcal{X} \rightarrow \mathcal{Y}$ which can predict
$Y$ from $X$.

A criterion is need to choose the function $g$. It can be
chosen as $P(g(X) \neq Y)$. We thus define the \textit{risk} of $g$ as
\begin{equation}
	R(g) = 	P(g(X) \neq Y) = \mathbb{E}[\mathbb{I}_{g(X)\neq Y}]
\end{equation}

Note that $P$ can be decomposed as $P_X \times P(Y|X)$. Define the \textit{regression function} $\eta(x) = \mathcal{E}[Y|X=x] = 2\mathcal{P}[Y=1|X=x] - 1$ and the
\textit{target function} (or Bayes classifier) $t(x) = sgn \eta(x)$. This function achieves the minimum risk over all
possible measurable functions:
\begin{equation}
	R(t) = \inf_g R(g)
\end{equation}
Denote the value $R(t)$ as $R^*$, called the Bayes risk.

Define the \textit{noise level} as $s(x) = \min(\mathbb{P}[Y=1|X=x], 1 - \mathbb{P}[Y=1|X=x]) = (1-\eta(x))/2$ and this gives $R^* = \mathbb{E}s(X)$.

Our goal is thus to find this function $t$. but since $P$ is unknown we cannot directly measure the risk. Let's define \textit{empirical risk}:
\begin{equation}
	R_n(g) = \frac{1}{b} \sum_{i=1}^n \mathbb{I}_{g(X_i)\neq Y_i}
\end{equation} 
It is common to use this quantity as a criterion to select an estimate of $t$.


\subsection{Algorithms}

\textbf{Overfitting}. When the input space is infinite, one can always construct a function $g_n$ which perfectly predicts the labels of the training data but behaves on the other points as the opposite of the target function $t$. So one would have minimum empirical risk but maximum risk.

There are essentially two ways to prevent this overfitting situation:
\begin{itemize}
\item restrict the class of functions in which the minimization is performed 
\item modify the criterion to be minimized (e.g. adding a penalty for complicated function)
\end{itemize}

\textbf{Empirical Risk Minimization}. This algorithm is one of the most straightforward, yet it is usually efficient:
\begin{equation}
	g_n = arg \min_{g\in \mathcal{G}} R_n(g)
\end{equation}
where $\mathcal{G}$ is the function class we perform minimization.

Sometime (actually in most case) the \textit{best} function in $\mathcal{G}$ is not necessary the overall best function. So one may want to enlarge the model as much as possible, while preventing overfitting.

\textbf{Structural Risk Minimization}.
An infinite sequence $\{\mathbf{G_d} | d=1, 2, \ldots \}$ of models of increasing size:
\begin{equation}
	g_n = arg \min_{g\in \mathcal{G}_d, d\in\mathcal{N}} R_n(g) + \text{pen}(d,n)
\end{equation}

The penalty $\text{pen}(d,n)$ gives preference to models where estimation error is small and measures the size of capacity of the model.



\textbf{Regularization}. Another, usually easier to implement approach consists in choosing a large model $\mathcal{G}$ and define on $\mathcal{G}$ a \textit{regularizer}, typically a norm $\|g\|$:
\begin{equation}
	g_n = arg \min_{g\in \mathcal{G}} R_n(g) + \lambda \|g\|^2
\end{equation}

Most existing (and successful) methods can be thought of
as regularization methods.


\textbf{Normalized Regularization}.

\subsection{Bounds} A learning algorithm takes as input the data $(X_1, Y_1), \ldots, (X_n, Y_n)$ and produce a function $g_n$ which depends on this data. We want to estimate the risk of $g_n$. However, $R(g_n)$ is a random variable and it cannot be computed from the data (since it also depends on the unknow $P$). Estimates of $R(g_n)$ will take the form of probabilistic bounds.

Suppose the algorithm chooses its output from a model $\mathcal{G}$, let $g^*$ be the best function in $\mathcal{G}$ with $R(g^*) = \inf_{g\in\mathcal{G}R(g)}$, we have
\begin{equation}
	R(g_n) - R^* = [R(g^*) - R^*] + [R(g_n) - R(g^*)]
\end{equation}

The first term on the right hand side is usually called the approximation error, and measures how well can functions in $\mathcal{G}$ could approach the target (if would be zero if $t\in \mathcal{G}$). The second terms, called estimation error is a random quantity (depends on the data) and measures how close is $g_n$ to the best possible choice in $\mathcal{G}$.

The first term is usually difficult to decide, it requires us to make some assumption on the model and data. We will focus on the estimation error

Another possible decomposition of the risk is the following:
\begin{equation}
	R(g_n) = R_n(g_n) + [R(g_n) - R_n(g_n)]
\end{equation}

Summary: three type of results we may be interested in.
\begin{itemize}
\item $R(g_n) \leq R_n(g_n) + B(n,\mathcal{G})$
\item $R(g_n) \leq R(g^*) + B(n,\mathcal{G})$
\item $R(g_n) \leq R^* + B(n,\mathcal{G})$
\end{itemize}




\section{Basic Bounds}
In this section, simple error bounds (also called generalization bounds) will be covered.

\subsection{Relationship to Empirical Processes}
TOne way to make a statement about $R(g_n)$ is to say how it relates to an estimate such as the empirical risk $R_n(g_n)$. This relationship can take the form of upper and lower bounds for
\begin{equation}
	\mathbb{P}[R(g_n) - R_n(g_n) > \epsilon]
\end{equation}

Let's denote $Z_i = (X_i,Y_i)$ and $Z = (X,Y)$. given $\mathcal{G}$ define the loss class
\begin{equation}
	\mathcal{F} = \{f: (x,y) \mapsto \mathbb{I}_{g(x)\neq y}: g\in \mathcal{G} \}
\end{equation}

Some shorthand notation: $Pf=\mathbb{E}[f(X,Y)]$ and $P_nf=\frac{1}{n}\sum_{i=1}^nf(X_i,Y_i)$. $P_n$ is usually called the empirical measure associated to the training sample. Our interest if the difference between true and empirical risks:
\begin{equation}
	\label{diff_Pf_Pnf}
	P f_n - P_n f_n
\end{equation}
An empirical process is a collection of random variable indexed by a class of functions, and such that each random variable is distributed as a sum of i.i.d. random variables:
\begin{equation}
	\{ P f - P_n f\}_{f\in\mathcal{F}} 
\end{equation}

One of the most studied quantity associated to empirical process is their supremum:
\begin{equation}
	\sup_{f\in\mathcal{F}}Pf-P_nf
\end{equation}
A upper bound on this quantity will also be a upper bound on (\ref{diff_Pf_Pnf}). This shows that the theory of empirical process is great source of tools and techniques for Statistical Learning Theory.

\subsection{Hoeffding's Inequality}
Let's rewrite $R(g) - R_n(g)$ as the following:
\begin{equation}
	R(g) - R_n(g) = \mathcal{E}[f(Z)] - \frac{1}{n}\sum_{i=1}^nf(Z_i)
\end{equation}

By the law of large numbers, we have:
\begin{equation}
	\mathbb{P}[\lim_{n\rightarrow \infty} \frac{1}{n}\sum_{i=1}^nf(Z_i) = \mathbb{E}[f(Z)]] = 1
\end{equation}
This indicates that with enough samples, the empirical risk of a function is a good approximation to its true risk.

\begin{thm} \textbf{(Hoeffding)}. 
Let $Z_1,\ldots, Z_n$ be i.i.d random variables with $f(Z)\in [a,b]$. Then for all $\epsilon > 0$, we have
\begin{equation}
	\mathbb{P}[\big |  \frac{1}{n}\sum_{i=1}^nf(Z_i) - \mathbb{E}[f(Z)]  \big | > \epsilon] \leq 2 \exp \bigg ( -\frac{2n\epsilon^2}{(b-a)^2} \bigg )
\end{equation}

\end{thm}

Denote the right hand side by $\delta$, solve $\epsilon$, we have
\begin{equation}
	\mathbb{P}\big[\big | P_n f - P f  \big | > (b-a) \sqrt{\frac{\log \frac{2}{\delta}}{2n}}\big] \leq \delta
\end{equation}

or with probability at least $1-\epsilon$, we have
\begin{equation}
	\big | P_nf - Pf  \big | \leq (b-a) \sqrt{\frac{\log \frac{2}{\delta}}{2n}}
\end{equation}

As $f(Z) \in [0, 1]$ in our case, we can conclude that with probability at least $1-\epsilon$
\begin{equation}
	R(g) \leq R_n(g) + \sqrt{\frac{\log \frac{2}{\delta}}{2n}}
\end{equation}

Notice that the function $g$ has to be fixed, if this function depends on the data then above inequality couldn't be applied directly. However, these sets $S$ may be different for different functions. In other words, for the observed sample, only some of the functions in $\mathcal{F}$ will satisfy this inequality.



\subsection{Limitations}
What above result essentially says is that for each (fixed) function $f\in \mathcal{F}$, there is a set $S$ of samples for which $Pf-P_nf \leq \sqrt{\frac{\log \frac{2}{\delta}}{2n}}$ (and the set of samples has measure $P[S] \geq 1 - \epsilon$)


\subsection{Uniform Deviations}

Before seeing the data, we do not know which function the algorithm will choose. The idea is to consider \textit{uniform} deviations
\begin{equation}
	R(f_n) - R_n(f_n) \leq \sup_{f\in \mathcal{F}} (R(f) - R_n(f))
\end{equation}

In other words, if we can upper bound the supremum on the right, we are done. For this we need a bound which holds simultaneously for all functions in a class.

Consider two functions $f_1$, $f_2$ and define
$$
	C_i = \{(x_1,y_1), \ldots , (x_n, y_n): Pf_i - P_nf_i > \epsilon\}
$$
by Hoeffding's inequality, for each $i$
$$
	\mathbb{P}[C_i] \leq \delta
$$
We want to measure how many samples are 'bad' for $i=1,2$.
$$
	\mathbb{P}[C_1\cup C_2] \leq \mathbb{P}[C_1] + \mathbb{P}[C_2] \leq 2\delta
$$

If $|\mathcal{F}| = N$, then
$$
	\mathbb{P}[C_1\cup \ldots \cup C_N] \leq \sum_{i=1}^N \mathbb{P}[C_i]
$$
As a result:
$$
	\mathbb{P}[\exists f\in \mathcal{F}: Pf-P_nf > \epsilon] \leq \sum_{i=1}^N \mathbb{P}[Pf_i-P_nf_i>\epsilon] \leq N \exp (-2n\epsilon^2)
$$

Now turn to $\mathcal{G} = {g_1,\ldots, g_N}$, for all $\delta > 0$ with probability at least $1-\delta$,
$$
	\forall g \in \mathcal{G}, R(g) \leq R_n(g) + \sqrt{\frac{\log N + \log\frac{1}{\delta}}{2n}}
$$

\subsection{Estimation Error}
Using the same idea as before, we can also get a bound on the estimation error. Let's start from the inequality
\begin{equation}
	R(g^*) \leq R_n(g^*) + \sup_{g\in \mathcal{G}}(R(g)-R_n(g))
\end{equation}
Further, we have 
$$
	R_n(g^*) - R_n(g_n) \geq 0
$$
Thus we obtain
\begin{align}
	R(g_n) &=     R(g_n) - R(g^*) + R(g^*)\\
		   &\leq  R_n(g^*) - R_n(g_n) + R(g_n) -R(g^*) + R(g^*)\\
		   &\leq  2\sup_{g\in\mathcal{G}}|R(g)-R_n(g)| + R(g^*)
\end{align}
We obtain that with probability at least $1-\delta$
\begin{equation}
	R(g_n) \leq R(g^*) + \sqrt{\frac{\log N + \log\frac{1}{\delta}}{2n}}
\end{equation}	
In the right hand side, both terms depend on the size of the class $\mathcal{G}$. If this size increases, the last first term will decrease, while the second will increase.




\section{Infinite Case: Vapnik-Chervonenkis Theory}
In this section, previous results will be extended to the case that the class $\mathcal{G}$ is infinite. This requires the introduction of tools from Vapnik-Chervonenkis Theory.

\subsection{Refined Union Bound and Countable Case}
Recall that by Hoeffding's inequality, for each $f\in \mathcal{F}$, for each $\delta > 0$ (possibly depending on $f$, which we write $\delta(f)$),
$$
	\mathbb{P} \bigg [ Pf-P_nf > \sqrt{\frac{\log \frac{1}{\delta(f)}}{2n}} \bigg] \leq \delta(f)
$$
With a countable set $\mathcal{F}$, the union bound yields
$$
	\mathbb{P} \bigg [\exists f\in \mathcal{F}: Pf-P_nf > \sqrt{\frac{\log \frac{1}{\delta (f)}}{2n}} \bigg ] \leq \sum_{f\in\mathcal{F}}\delta(f)
$$

Choosing $\delta (f) = \delta p(f)$ with $\sum_{f\in\mathcal{F}}p(f) = 1$, this makes the right hand side equal to $\delta$ and we have: with probability at least $1-\delta$,
\begin{equation}
	\forall f\in \mathcal{F}, Pf\leq P_nf + \sqrt{\frac{\log \frac{1}{p(f)} + \log \frac{1}{\delta}}{2n}}
\end{equation}

This approach provides a way to put knowledge about the algorithm into $p(f)$. However $p$ should be chosen before seeing the data. In general if $p$ is well-chosen, the bound will have a small value. 

\subsection{General Case}
We should also include the case that $\mathcal{G}$ is uncountable.
Given a sample $z_1,\ldots, z_n$, consider
$$
	\mathcal{F}_{z_1,\ldots,z_n} = \{ (f(z_1),\ldots,f(z_n)): f\in \mathcal{F}\}
$$

\begin{definition}\textbf{(Growth function)}. 
\begin{equation}
	S_{\mathcal{F}}(n) = \sup_{(z_1,\ldots,z_n)}|\mathcal{F}_{z_1,\ldots,z_n} |
\end{equation}

\end{definition}
It's trivial to see that $S_{\mathcal{F}}(n) = S_{\mathcal{G}}(n)$

\begin{thm}\textbf{(Vapnik-Chervonenkis).} 
	\label{v-c}
For any given $\delta > 0$, with the probability at least $1-\delta$,
$$
	\forall g\in\mathcal{G}, R(g) \leq R_n(g)+ 2\sqrt{2\frac{\log S_{\mathcal{G}}(2n) + \log \frac{2}{\delta}}{n}}
$$
\end{thm}
Note that, in the finite case, $|\mathcal{G}|=N$, we have $S_{\mathcal{G}}(n)\leq N$, so above bound is better than the one we had before. Now the most difficult problem will be how to compute $S_{\mathcal{G}}(n)$.


\subsection{VC Dimension}
\begin{definition} \textbf{(VC dimension)}
The VC dimension of a class $\mathcal{G}$ is the largest $n$ such that
$$
	S_{\mathcal{G}}(n) = 2^n
$$
\end{definition}

In other words, the VC dimension of a class $\mathcal{G}$ is the size of the largest set that it can shatter.

Many examples of VC dimension could be found from \cite{link5} \href{www.spsc.tugraz.at/system/files/vc_examples.pdf‎}{\textit{\textbf{link}}}.

\begin{lem}\textbf{(Vapnik and Chervonenkis, Sauer, Shelah).} Let $\mathcal{G}$ be a class of function with finite VC-dimension h. Then for all $n\in \mathbb{N}$,
$$
	S_{\mathcal{G}}(n) \leq \sum_{i=0}^h {n \choose i}
$$
and for all $n \geq h$,
$$
	S_{\mathcal{G}}(n) \leq \bigg ( \frac{en}{h} \bigg )^h
$$

\end{lem}

Use this lemma and theorem \ref{v-c}, we immediately  obtain that if $\mathcal{G}$ has VC dimension $h$, with probability at least $1-\delta$,
$$
	\forall g\in\mathcal{G}, R(g) \leq R_n(g)+ 2\sqrt{2\frac{h\log \frac{2en}{h}  + \log \frac{2}{\delta}}{n}}
$$

\subsection{VC Entropy}
VC dimension is \textit{distribution independent.} Hence, it allows to get bounds that do not depend on the problem
at hand, the same bound holds for any distribution. This is in some sense an advantage, but on the other hand, this bound may too loose for most distributions.

We will introduce a distribution-dependent result. Use the notation $N(\mathcal{F}, z_1^n) := |\mathcal{F}_{z_1,\ldots,z_n}|$.

\begin{definition} \textbf{(VC entropy)}
The (annealed) VC entropy is defined as
$$
	H_{\mathcal{F}}(n) = \log \mathbb{E}[N(\mathcal{F}, Z_1^n)].
$$
\end{definition}

\begin{thm}
For any $\delta > 0$, with probability at least $1-\delta$,
$$
	\forall g \in \mathcal{G}, R(g) \leq R_n(g)+ 2\sqrt{2\frac{ H_{\mathcal{G}}(2n) + \log \frac{2}{\delta}}{n}}
$$
\end{thm}





\section{Capacity Measures}
We have introduced three measures of \textit{capacity} or size of classes of function: the VC dimension and growth function both distribution independent and the VC entropy which depends on the distribution. Apart from the VC dimension, they are usually difficult or impossible to compute. In this section, we will introduce some other measures which not only may give sharper estimates, but also have properties that make their computation possible from the data only.
\subsection{Covering Numbers}
Define the normalized Hamming distance of the 'projections' on the sample.
$$
	d_n(f,f') = \frac{1}{n}|\{ f(Z_i) \neq f'(Z_i): i=1,\ldots,n \} |
$$
We say that a set $f_1,\ldots,f_N$ covers $\mathcal{F}$ at radius $\epsilon$ if
$$
	\mathcal{F} \subset \cup_{i=1}^N B(f_i,\epsilon)
$$
We then define the covering number of $\mathcal{F}$ as follows.	

\begin{definition} \textbf{(Covering number)}
The covering number of $\mathcal{F}$ at radius $\epsilon$, with respect to $d_n$, denoted by $N(\mathcal{F},\epsilon, n)$ is the minimum size of a cover of radius $\epsilon$
\end{definition}
Note that $N(\mathcal{F},\epsilon, n) = N(\mathcal{G}, \epsilon, n)$.

When the covering number are finite, it is possible to approximate the class $\mathcal{G}$ by a finite set of functions (which cover $\mathcal{G}$). 
\begin{thm}
	For any $t > 0$,
	$$
	\mathbb{P}[\exists g\in\mathcal{G}: R(g) > R_n(g)+t] \leq 8\mathbb{E}[N(\mathcal{G}, t, n)] e^{-nt^2/128}
	$$
\end{thm}



\subsection{Rademacher Averages}
Singed empirical measure $R_nf=\frac{1}{n}\sum_{i=1}^n\sigma_if(Z_i)$. Where $\sigma_i$ takes $-1$ or $1$ both with probability $0.5$.
 
\begin{definition} (Rademacher averages).
For a class $\mathcal{F}$ of functions, the Rademacher averages is defined as
$$
	\mathcal{R}(\mathcal{F}) = \mathbb{E} \sup_{f\in\mathcal{F}}R_nf
$$
\end{definition}


\begin{thebibliography}{99} % don't worry about the 99

\bibitem{paper1}
Bousquet, Olivier, Stéphane Boucheron, and Gábor Lugosi. \textbf{\textit{Introduction to statistical learning theory.}} Advanced Lectures on Machine Learning. Springer Berlin Heidelberg, 2004. 169-207.


\bibitem{Book2}
Vapnik, Vladimir N. \textbf{\textit{Statistical learning theory.}} (1998).

\bibitem{paper3}
Mark D. Reid.
\textbf{\textit{Generalization Bounds.}}
Encyclopedia of Machine Learning, November, 2010.

\bibitem{Book4}
MacKay, David JC. \textbf{\textit{Information theory, inference and learning algorithms.}} Cambridge university press, 2003.

\bibitem{link5}
\textit{\textbf{www.spsc.tugraz.at/system/files/vc\_examples.pdf‎}}

\bibitem{notes}
Banerjee, Arindam. \textit{\textbf{Lecture notes on Machine Learning Theory}}. University of Minnesota, Fall 2013.


\end{thebibliography}

\end{document}
